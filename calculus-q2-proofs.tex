\documentclass[11pt]{article}
\usepackage{fontspec}
\usepackage{amsmath}
\usepackage{amsfonts}
\usepackage{amssymb}
\usepackage[letterpaper,margin=1in]{geometry}
\usepackage[hidelinks]{hyperref}
\usepackage[nameinlink]{cleveref}
\usepackage{fancyhdr}
\usepackage{cancel}

\title{Calculus Q2 Proofs}
\author{Adam Zhang}

\newcommand{\nimplies}{\mathrel{{\ooalign{\hidewidth$\not\phantom{=}$\hidewidth\cr$\implies$}}}}
% Primed vectors are not typeset correctly and I don't feel like properly fixing
% it, so let's just use bold vectors instead.
\renewcommand{\vec}[1]{\ensuremath\mathbf{#1}}

\crefname{proof}{Proof}{Proofs}
\crefalias{enumi}{proof}

\begin{document}
\pagestyle{fancy}
\fancyhead[L]{AET Multivariable Calculus\\\textbf{Quarter 2 Proofs}}
\fancyhead[R]{Adam Zhang\\}

\begin{center}
  \emph{On my honor, I will not accept nor provide any unauthorized aid on this assignment.}
\end{center}

\section*{Proofs}

\begin{enumerate}
\item If \(\vec{r}(t) = \vec{a}\cos(\omega t) + \vec{b}\sin(\omega t)\), where \(\vec{a}\) and \(\vec{b}\) are constant vectors, show that: \(\vec{r}(t) \times \vec{r} '(t) = \omega \vec{a} \times \vec{b}\) (\textbf{Note}: \(\vec{r}(t)\) is not in component form).
  \begin{align*}
    \vec{r}'(t) &= -\omega \vec{a}\sin(\omega t) + \omega \vec{b} \cos (\omega t) \\
    \vec{r}(t) \times \vec{r}'(t) &= \left( \vec{a} \cos(\omega t) + \vec{b}
                                    \sin(\omega t) \right) \times \left( -\omega
                                    \vec{a} \sin(\omega t) + \omega \vec{b}
                                    \cos(\omega t) \right) \\
                &= \vec{a}\cos(\omega t) \left( \vec{a} \times \omega \vec{b} \right) +
                  \sin(\omega t) \sin(\omega t) \left( \vec{b} \times -\omega \vec{a}
                  \right) \\
  \end{align*}
\item If \(\vec{u}(t) = \vec{r}(t) \cdot \left[ \vec{r}'(t) \times
    \vec{r}''(t) \right]\), show that \(\vec{u}'(t) = \vec{r}(t) \cdot \left[
    \vec{r}'(t) \times \vec{r}'''(t) \right]\).
  \begin{align*}
    \vec{u}'(t) &= \cancel{\vec{r}'(t) \cdot \left[ \vec{r}'(t) \times \vec{r}''(t)
                  \right]} + \vec{r}(t) \cdot \big[ \cancel{\vec{r}''(t) \times \vec{r}''(t)} +
                  \vec{r}'(t) \times \vec{r}'''(t) \big] \\
                &= \vec{r}(t) \cdot \left[ \vec{r}'(t) \times \vec{r}'''(t) \right]
  \end{align*}
\item Show that \(\vec{A} \cdot \frac{\mathrm{d}\vec{A}}{dt} = \lVert
  \vec{A} \rVert \frac{\mathrm{d} \lVert \vec{A} \rVert}{\mathrm{d}t}\).

  Let \(\vec{A} = \vec{r}(t)\) and \(m(t) = \lVert \vec{r}(t) \rVert\).
  \begin{align*}
    \vec{r}(t) \cdot \frac{\mathrm{d}\vec{r}}{\mathrm{d}t}
    &= m(t) \frac{\mathrm{d}m}{dt} \\
    &= \frac{\mathrm{d}}{\mathrm{d}t} \int m(t)m'(t) \mathrm{d}t \\
    &= \frac{\mathrm{d}}{\mathrm{d}t} \left[ \frac{m(t)^2}{2} \right] \\
    &= \frac{1}{2} \frac{\mathrm{d}}{\mathrm{d}t} \left[ \vec{r}(t) \cdot
      \vec{r}(t) \right] \\
    &= \frac{1}{2} \left( \frac{\mathrm{d}\vec{r}}{\mathrm{d}t} \cdot
      \vec{r}(t) + \frac{\mathrm{d}\vec{r}}{\mathrm{d}t} \cdot \vec{r}(t) \right) \\
    &= \vec{r}(t) \cdot \frac{\mathrm{d}\vec{r}}{\mathrm{d}t}
  \end{align*}

\item You are told that there is a function \(f(x, y)\) whose partial
  derivatives are given by \(f_x(x, y) = -2xy \sin(x^2y)\) and \(f_y(x, y) =
  -x^2 \sin(x^2y)\). Should you believe it? Why or why not?
  \begin{align*}
    \int f_x \,\mathrm{d}x &= -\int 2xy \sin(x^2y) \,\mathrm{d}x \\
                           &= \cos(x^2y) + g(y) \\
    \int f_y \,\mathrm{d}y &= -\int x^2 \sin(x^2y) \,\mathrm{d}y \\
                           &= \cos(x^2y) + h(x) \\
    \therefore f(x, y) &= \cos(x^2y)
  \end{align*}

\item The gas law for a fixed mass \(m\) of an ideal gas at absolute
  temperature \(T\), pressure \(P\), and volume \(V\) is \(PV = mRT\) where
  \(R\) is the gas constant. Show that \(T \frac{\partial V}{\partial T}
  \frac{\partial P}{\partial T} = mR\).

  \begin{align*}
    V = \frac{mRT}{P} \land P = \frac{mRT}{V} \land T = \frac{PV}{mR} \\
    T \frac{\partial V}{\partial T}
    \frac{\partial P}{\partial T} = \frac{\cancel{PV}}{\cancel{mR}} \frac{\cancel{mR}}{\cancel{P}} \frac{mR}{\cancel{V}} = mR
  \end{align*}

\item Prove, given that the third-order partial derivatives of \(f(x, y,
  z)\) are continuous then, \(f_{xyz} = f_{zxy} = f_{yzx}\).

  If \(f(x) : \mathbb{R}^n \mapsto \mathbb{R}\) has continuous second
  derivatives at \(x\), then \(\frac{\partial^2 f}{\partial x_ix_j} =
  \frac{\partial^2 f}{\partial x_jx_i}\) (Clairaut's theorem).

\item Verify that if \(z(x, y) = f(x^2 + y^2)\) then \(y \frac{\partial
    z}{\partial x}(x,y) - x \frac{\partial z}{\partial y}(x,y) = 0\).

  \begin{align*}
    \frac{\partial z}{\partial x}
    &= f'{\left(x^2 + y^2\right)} \cdot 2x = 2x
      f'{\left(x^2 + y^2\right)} \\
    \frac{\partial z}{\partial y}
    &= f'{\left( x^2 + y^2 \right)} \cdot 2y = 2y f'{\left( x^2 + y^2 \right)}
    \\
    y \frac{\partial z}{\partial x}(x,y) - x &\frac{\partial z}{\partial
                                               y}(x,y) = \cancel{2xy f'{\left( x^2 + y^2 \right)}} - \cancel{2xy f'{\left( x^2 + y^2
                                               \right)}} = 0
  \end{align*}

\item Let \(\alpha > 0\) be a constant. Prove that \(u(x,y,z,t) =
  \frac{1}{t^{\frac{3}{2}}} e^{-\frac{x^2 + y^2 + z^2}{4\alpha t}}\), satisfies that
  heat equation given by: \(u_t = \alpha\left(u_{xx} + u_{yy} + u_{zz}\right)\) for all
  \(t > 0\).

  Let \(E = e^{-\frac{x^2 + y^2 + z^2}{4\alpha t}}\). Due to the symmetry of
  \(u(x,y,z,t)\) across \(x\), \(y\), and \(z\), \eqref{p8:uxx} can be used
  to derive \eqref{p8:uyy} and \eqref{p8:uzz}.

  \begin{subequations}
    \begin{align*}
      u_x(x,y,z,t) &= \frac{1}{t^{\frac{3}{2}}} E \left( -\frac{x}{2\alpha t}
                     \right) = -\frac{x}{2\alpha t^{\frac{5}{2}}} E \\
      u_{xx}(x,y,z,t) &= -\frac{1}{2\alpha t^{\frac{5}{2}}} E +
                        -\frac{x}{2\alpha t^{\frac{5}{2}}} E \left( -\frac{x}{2\alpha t} \right)
      \\
                   &= E \left( -\frac{1}{2\alpha t^{\frac{5}{2}}} + \frac{x^2}{4 \alpha^2
                     t^{\frac{7}{2}}} \right)
                     \stepcounter{equation}\tag{\theequation}\label{p8:uxx} \\ 
      u_{yy} &= E \left( -\frac{1}{2\alpha t^{\frac{5}{2}}} +
               \frac{y^2}{4\alpha^2 t^{\frac{7}{2}}} \right)
               \stepcounter{equation}\tag{\theequation}\label{p8:uyy} \\
      u_{zz} &= E \left( -\frac{1}{2\alpha t^{\frac{5}{2}}} +
               \frac{y^2}{4\alpha^2 t^{\frac{7}{2}}} \right)
               \stepcounter{equation}\tag{\theequation}\label{p8:uzz} \\
    \end{align*}
  \end{subequations}
  \(u_t(x,y,z,t)\) is given by \eqref{p8:ut}.

  \begin{equation}
    \label{p8:ut}
    \begin{split}
      u_t(x,y,z,t) &= -\frac{3}{2t^{\frac{5}{2}}} E + \frac{1}{t^{\frac{3}{2}}}
                     E \left( \frac{x^2 + y^2 + z^2}{4\alpha t^2} \right) \\
                   &= E \left( -\frac{3}{2t^{\frac{5}{2}}} + \frac{x^2 + y^2 + z^2}{4\alpha
                     t^{\frac{7}{2}}} \right)
    \end{split}
  \end{equation}

  Combine to show equality.

  \begin{align*}
    E \left( -\frac{3}{2t^{\frac{5}{2}}} + \frac{x^2 + y^2 + z^2}{4\alpha
    t^{\frac{7}{2}}} \right) &= \alpha \left( u_{xx} + u_{yy} + u_{zz}
                               \right) \\
                             &= \alpha E \left( -\frac{3}{2\alpha t^{\frac{5}{2}}} + \frac{x^2 + y^2
                               + z^2}{4\alpha^2 t^{\frac{7}{2}}} \right) \\
                             &= E \left( -\frac{3}{2 t^{\frac{5}{2}}} + \frac{x^2 + y^2 +
                               z^2}{4\alpha t^{\frac{7}{2}}} \right)
  \end{align*}

\end{enumerate}

\newpage
\section*{True or False}
Prove that the given statement is true in \(\mathbb{R}^3\) or provide a counterexample to show that it is false.

\begin{enumerate}
  \setcounter{enumi}{8}
\item The curves \(\vec{r}_1(t) = (-1 + t)\hat{\imath} + (1 + 2t)\hat{\jmath} +
  (5 - t)\hat{k}\) and \(\vec{r}_2(t) = (2 + 2t)\hat{\imath} + (4 +
  t)\hat{\jmath} + (3 + t)\hat{k}\) intersect.

  Let \(\vec{r}_2(s) = (2 + 2s)\hat{\imath} + (4 + s)\hat{\jmath} + (3 +
  s)\hat{k}\).

  \begin{equation*}
    \begin{gathered}
      (-1 + t)\hat{\imath} = (2 + 2s)\hat{\imath} \implies t = 2s + 3 \\
      (1 + 2t)\hat{\jmath} = (4 + s)\hat{\jmath} \\
      s + 4 = 1 + 4s + 6 \\
      3s = -3 \\
      s = -1 \land t = 1 \\
      (5 - t)\hat{k} = (3 + s)\hat{k} \\
      4 \neq 2 \\
      \therefore \not\exists(t, s) \in \mathbb{R}^2 : \vec{r}_1(t) = \vec{r}_2(s)
    \end{gathered}
  \end{equation*}

\item If \(\vec{r}(t)\) is a differentiable vector function, then
  \(\frac{\mathrm{d}}{\mathrm{d}t} \lVert \vec{r}(t) \rVert = \lVert
  \vec{r}'(t) \rVert\). \label{proof10}

  Let \(\vec{r}(t) = \cos(t^2)\hat{\imath} + \sin(t^2)\hat{\jmath}\).

  \begin{align*}
    \forall t \in \mathbb{R} : \lVert \vec{r}(t) \rVert
    &= \sqrt{\cos^2(t^2) + \sin^2(t^2)} = 1 \\
    &\therefore \frac{\mathrm{d}}{\mathrm{d}t} \lVert \vec{r}(t) \rVert = 0 \\
    \vec{r}'(t)
    &= -2t \sin(t^2)\hat{\imath} + 2t\cos(t^2)\hat{\jmath} \\
    \lVert \vec{r}'(t) \rVert
    &= \sqrt{4t^2\sin^2(t^2) + 4t^2\cos^2(t^2)} = 2t \\
    \exists t \in \mathbb{R} : 0 \neq 2t
    &\therefore \exists(\vec{r}(t) : \mathbb{R} \mapsto \mathbb{R}^3) \exists
      t \in \mathbb{R} : \frac{\mathrm{d}}{\mathrm{d}t} \lVert \vec{r}(t)
      \rVert \neq \lVert \vec{r}'(t) \rVert
  \end{align*}

\item If \(\lVert \vec{r}(t) \rVert = 1\) for all \(t\), then \(\lVert
  \vec{r}'(t) \rVert\) is a constant.

  Let \(\vec{r}(t) = \cos(t^2)\hat{\imath} + \sin(t^2)\hat{\jmath}\). By
  \cref{proof10}, \(\forall t \in \mathbb{R} : \lVert \vec{r}(t) \rVert = 1
  \land \lVert \vec{r}'(t) \rVert = 2t \therefore \lVert \vec{r}(t) \rVert =
  1 \nimplies \forall t \in \mathbb{R} : \lVert \vec{r}'(t) \rVert = c\).

\item Any surface which is the level surface of a three-variable function
  \(g(x, y, z)\) can also be represented as the graph of a two-variable
  function \(f(x, y)\).

  Let \(g(x, y, z) = x^2 + y^2 + z^2\). The level surface \(g(x, y, z) =
  1\) is given by \(x^2 + y^2 + z^2 = 1\). Solving for \(z\) yields \(z =
  \pm \sqrt{1 - x^2 + y^2}\), which cannot be represented as a function,
  as the range contains multiple outputs per input.

\item If \(f\) is a function, then \(\lim_{(x, y, z) \to (2, 5, 3)}
  f(x, y, z) = f(2, 5, 3)\).

  \begin{equation*}
    \begin{gathered}
      f(x, y, z) : \mathbb{R}^3 \mapsto \mathbb{R} =
      \begin{cases}
        x + y + z & \text{if } x \geq 2 \\
        x + y + 1 & \text{if } x < 2
      \end{cases} \\
      f(2, 5, 3) = 10 \land \lim_{(x, y, z) \to (2, 5, 3)} f(x, y, z)
      \text{ does not exist}
    \end{gathered}
  \end{equation*}
\end{enumerate}

\end{document}

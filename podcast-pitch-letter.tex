\documentclass[12pt]{article}
\usepackage[letterpaper,margin=1in]{geometry}
\usepackage{hyperref}
\usepackage{parskip}
\usepackage{titlesec}

\frenchspacing

\renewcommand\thesection{Episode \arabic{section}: }
\titlelabel{\thetitle}

\begin{document}
\begin{flushright}
  19019 Upper Belmont Place \\ Leesburg, VA 20176
  
  \medskip \today
\end{flushright}

\medskip
Dear Mrs. Jennings’s Block 5 Class,

\medskip
I am excited to reveal to you my podcast idea for the People’s Choice Podcast
Awards, Behind the Bits. Behind the Bits is a podcast series focused on the
modern loss of digital sovereignty, the predatory power relinquished to
monopolistic technology corporations, and the resulting societal harms. The
modern emphasis on “profit at all costs” has resulted in multifaceted harm to
consumers’ rights, such as sovereignty (ownership over our own data), privacy,
and digital freedoms. Behind the Bits aims to be an outlet to communicate these
issues to the general public, along with both personal and possible societal
remedies for these issues.

\section{You Are The Product}
My first episode, “You Are The Product”, will serve as the foundational context for the rest of the season. In it, I want to discuss the economic structure of Western society, and how it feeds into the predatory market practices within the technology sector [citation needed]. I want to establish the insane profits that companies such as Google/Meta make, and where the money comes from [citation needed]—especially since these companies provide services that are free of monetary charge to consumers.

Episode 2: Nothing to Hide
In this episode, I want to discuss the problematic aspects of how companies turn personal information into a tradable asset, and the resulting privacy violations. The title comes from the argument “why should I care about privacy if I have nothing to hide”, a common fallacious argument I want to quickly debunk [citation needed]. I want to establish that companies have no care for the safety of users’ information, and actively exploit the trust of users to increase their profits.

Episode 3: Algorithmic Pricing [note: I don’t like this title]

Episode 4: ??? [this is not a title, I just don’t have a filler here yet]

Episode 5: Your Data, Your Hands

Episode 6: Societal Shifts [note: I don’t like this title]

By this point you should now be revealing specific information about your first season. What do you hope to accomplish overall? Is there a common thread? What are you hoping to achieve within each episode? This section of the letter will be long – you will need to decide how you will keep your audience’s attention as you list the six episodes for the season AND their descriptions (each description should be no fewer than five specific sentences that reveal the overall content/message/etc.). How will you divide this information to maintain your reader’s attention? At the end of each episode synopsis, you should have the link(s) to the research used for that potential episode. EACH of the SIX episodes should have a title in quotation marks. You will want to look at sample podcasts from a structural and visual standpoint and explain your podcast plan accordingly alongside the research. While you describe your episodes, you will use in-text citations to credit your sources.

This next paragraph will be shorter since its purpose is to project a new season. As you indicate your hope for a second season, give an overview of what you hope to accomplish by having a new season. Again, you will need to make sure that you have episode ideas and descriptions and links (but not as many as the first season’s plan).

Contact
I appreciate your time and consideration of my podcast proposal. For more details or for any other points of discussion, please feel free to contact me at fakeemail@notrealdomain.com or at +1 123-456-7890.

\end{document}

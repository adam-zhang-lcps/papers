\documentclass[12pt]{article}
\usepackage[letterpaper,margin=1in]{geometry}
\usepackage{xcolor}
\usepackage{hyperref}
\usepackage{parskip}
\usepackage{titlesec}
\usepackage[backend=biber,style=mla]{biblatex}

\addbibresource{refs.bib}

\frenchspacing

\titleformat{\section}
{\normalfont\bfseries} % Bold font
{Episode \thesection:} % No numbering prefix (e.g., "Section" text)
{1ex}                  % Space between prefix and title
{}                     % Style before title
  
\titlespacing*{\section}
{0pt}
{\bigskipamount}
{0pt}

\hypersetup{
  colorlinks=true,
  linkcolor=blue,
  linkbordercolor=blue,
  urlcolor=blue,
  urlbordercolor=blue,
  citecolor=black,
  citebordercolor=white,
}

% Workaround for `colorlinks=true` disabling borders with no option to re-enable them.
\makeatletter
\Hy@AtBeginDocument{%
  \def\@pdfborderstyle{/S/U/W 1}% Overrides border style set with colorlinks=true
}
\makeatother

\newcommand{\name}{\emph{Behind the Bits}}

\begin{document}
\begin{flushright}
  19019 Upper Belmont Place \\ Leesburg, VA 20176
  
  \medskip \today
\end{flushright}

\medskip
Dear Mrs. Jennings’s Block 5 Class,

\medskip I am excited to reveal to you my podcast idea for the People’s Choice
Podcast Awards, \name. \name{} is a brand new podcast series focused on the
modern loss of digital sovereignty, the predatory power relinquished to
monopolistic technology corporations, and the resulting personal and societal
harms. The modern emphasis on ``profit at all costs'' has resulted in
multifaceted harm to consumers’ rights, such as sovereignty (ownership over our
own data), privacy, and digital freedom. \name{} aims to be both an educational
outlet to communicate these hidden issues to the general public, along with
advocating for both personal and possible long-term societal remedies.

\section{You Are The Product}
My first episode, ``You Are The Product'', will serve as the introduction of the
broad topics I will be covering, providing foundational context for the rest of
the season. In it, I want to discuss the economic structure of Western society,
and how it both allows and feeds into the predatory market practices within the
technology sector \autocite{Elvy2017:PayingForPrivacyAndThePersonalDataEconomy}.
I want to establish the insane profits that companies such as Google/Meta make,
and where the money comes from [citation needed]—especially since these
companies provide services that are free of monetary charge to consumers.

\section{Nothing to Hide}
In this episode, I want to discuss the problematic aspects of how companies turn
personal information into a tradable asset, and why the resulting privacy
violations are a real threat.

The title comes from the argument ``why should I care about privacy if I have
nothing to hide'', a common fallacious argument I want to quickly debunk.
Privacy, after all, is about more than ``I'm not doing anything illegal''. It's
about your ability to be free, to be an independent thinker. It's about feeling
safe in your own life. In the end, we all have something to hide; maybe not from
the government, but certainly from each other---such is the essence of personal
living \autocite{Mordini2008:NothingHideBiometricPrivacyPrivateSphere}.

I want to establish that companies have no care for the safety of users’
information, and actively exploit the trust of users to increase their profits.

\section{Your Data, Their Decisions}
In 2022, a dad lost over a decade's worth of photos, videos, emails,
\emph{memories}, for merely taking a photo of his baby child's swollen genitalia to
send to a doctor. What he lost that day was access to the Google account he had
relied upon for countless years, all because Google's automated systems flagged
him for child abuse material. Now, child abuse is abhorrent and should not be
tolerated in the slightest, but should that protection of children come at the
expense of our control over our own digital lives? In fact, the situation gets
worse. Even after a police investigation, Google refused to restore access to
the account. All his data was gone, just like that \autocite{Hill2022:DadTookPhotosHisNakedToddlerDoctor}.

In ``Your Data, Their Decisions'', I aim to take a deep dive directly into the
most fundamental rabbit hole---our data sovereignty in the digital age. Who
really owns \emph{our} data? Well, the answer might very well be the services
you rely upon every day, who, as we've seen, will turn their back on you the
moment it's no longer profitable to allow you to use their services. For
example, many social media platforms claim near-total control over any content
you upload to their sites in their Terms of Service
\autocite{Tanner2022:IAgreedWhat}. Yeah, that 100-page document that I'm sure
you didn't read---after all, I didn't either. This has become evermore prevalent
in the era of artificial intelligence, where foundational companies such as
OpenAI, Meta, Google, and Anthropic are foaming at the mouths to get their hands
on as much user data as possible to train their models---and they're willing to
pay for it. Perhaps that's why even the FTC is calling out these abuses of the
Terms of Service as ``predatory'' and ``deceptive''
\autocite{Crystal2024:AiCompanies}.

This is just the tip of the iceberg. It's time to find out just how deep the
rabbit hole goes.

\section{The Price Isn't Right}
Every episode thus far has discussed, in a sense, ``theoretical'' problems.
Problems that, while certainly present, perhaps aren't visible in our
daily lives. After all, technology companies have done a very good job of hiding
them away from us---out of sight, out of mind. In this episode, I want to shine
the light on two direct harms resulting from abuse by technology companies:
social media psychology and algorithmic pricing.

It is perhaps well-known at this point that social media can be harmful to young
children and teens. Among adolescents, social media has resulted in less
face-to-face interaction, a greater desire to ``fit in'' with peers, and
increased rates of depression and other mental health issues \autocite{Allen2019:Social}.

Algorithmic pricing refers to any case in which algorithms directly control
prices for consumers. There are reasonable uses for algorithmic pricing; for
example, gasoline and the stock market. However, the harm truly begins when
algorithmic pricing becomes \emph{personalized} pricing; using data about you to
determine how much you're willing to pay. The more they think you'll pay, the
more they're going to charge you \autocite{Bar-Gill2019:Symposium}. Oh, but the
price will never go down---all that matters is how much \emph{more} money they
can extract from your pockets. Is this acceptable? It's time to start caring
about digital privacy.

\section{Take Control}

\section{Seize the Means}


\section*{Looking Ahead}

\section*{Contact}
I appreciate your time and consideration of my podcast proposal. For more
details or for any other points of discussion, please feel free to contact me at
\href{mailto:contact@behindthebits.com}{contact@behindthebits.com} or at +1 123-456-7890.

\clearpage
\begin{center}
  Sources List
\end{center}
\printbibliography[title={\relax}]

\end{document}

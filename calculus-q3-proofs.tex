\documentclass[11pt]{article}
\usepackage{fontspec}
\usepackage{amsmath}
\usepackage{amsfonts}
\usepackage{amssymb}
\usepackage[letterpaper,margin=1in]{geometry}
\usepackage[hidelinks]{hyperref}
\usepackage[nameinlink]{cleveref}
\usepackage{fancyhdr}
\usepackage{cancel}

\title{Calculus Q3 Proofs}
\author{Adam Zhang}

\newcommand{\nimplies}{\mathrel{{\ooalign{\hidewidth$\not\phantom{=}$\hidewidth\cr$\implies$}}}}
% Primed vectors are not typeset correctly and I don't feel like properly fixing
% it, so let's just use bold vectors instead.
\renewcommand{\vec}[1]{\ensuremath\mathbf{#1}}

\crefname{proof}{Proof}{Proofs}
\crefalias{enumi}{proof}

\setlength\parindent{0pt}

\begin{document}
\pagestyle{fancy}
\fancyhead[L]{AET Multivariable Calculus \\ \textbf{Quarter 3 Proofs}}
\fancyhead[R]{Adam Zhang \\}

\begin{center}
  \emph{On my honor, I will not accept nor provide any unauthorized aid on this assignment.}
\end{center}

\section*{Proofs}
\begin{enumerate}
\item Assume that all the given functions have continuous second-order partial
  derivatives. Show that any function of the form \(z = f(x + at) + g(x - at) \)
  is a solution of the wave equation \(\frac{\partial^2 z}{\partial t^2} = a^2
  \frac{\partial^2 z}{\partial x^2}\) (\textit{Hint}: Let \( u = x + at \), \( v
  = x - at \)).

\item Suppose that directional derivatives of \(f(x,y)\) are known at a point in two nonparallel directions given by unit vectors \(\vec{u} = \langle a, b \rangle\) and \(\vec{v} = \langle c, d \rangle\). Let the directional derivatives be defined as \(D_{u} f\) and \(D_{v} f\). Is it possible to find \( \nabla f \) at this point? If so, find \( \nabla f \) in terms of \( D_{u} f \), \( D_{v} f \), and the unit vectors \(\vec{u} = \langle a, b \rangle\) and \(\vec{v} = \langle c, d \rangle\).

\item Are there any points on the hyperboloid \(x^2 - y^2 - z^2 = 1\) where
  the tangent plane is parallel to the plane \(z = x + y\)? If yes, find
  the point(s). If no, clearly explain why.

\item Use Lagrange Multipliers to prove that the triangle with maximum area that
  has a given perimeter \(p\) is equilateral.

  \textit{Hint}: Use Heron's formula for the area:
  \(A = \sqrt{s(s - x)(s - y)(s - z)}\) where \(s = \frac{p}{2}\) and
  \(x, y, z\) are the lengths of the sides.

\item Prove that \( \nabla (FG) = F \nabla G + G \nabla F \) where \( F \) and
  \( G \) are differentiable scalar functions of \( x, y \) and \( z \).

\item If \(f(x,y)\) is continuous on \([a,b] \times [c,d]\) and
  \(g(x,y) = \int_0^x \int_0^y f(s,t) \, \mathrm{d}t \, \mathrm{d}s\) for
  \(a < x < b, c < y < d\), show that \(g_{xy} = g_{yx} = f(x,y)\).

\item Prove Property 10 from page 1059 in Section 15.2.
  \begin{quote}
    If \(m \leq f(x,y) \leq M\) for all \((x,y)\) in \(D\), then
    \(m \cdot A(D) \leq \iint_D f(x,y) \, \mathrm{d}A \leq M \cdot A(D)\).
  \end{quote}
\end{enumerate}

\section*{True or False}
Prove that the given statement is true or provide a counterexample to show that it is false.

\begin{enumerate}
  \setcounter{enumi}{7}

\item If \(f\) has a local minimum at \((a,b)\) and \(f\) is differentiable at
  \((a,b)\), then \(\nabla f (a,b) = 0\).

\item If \( f(x,y) \) has two local maxima, then \( f \) must have a local minimum.
\end{enumerate}

\textit{For problems \#10--12, if the answer is false, your explanation must include what the correct integral should be.}

\begin{enumerate}
  \setcounter{enumi}{9}
  
\item
  \(\int_{-\frac{1}{2}}^{2} \int_{\frac{y^2}{2}}^{\frac{3}{2}y+1} x^2 \sin(x-y)
  \,\mathrm{d}x\,\mathrm{d}y= \int_0^4 \int_{\frac{2}{3}(x-1)}^{\sqrt{x}} x^2
  \sin(x-y) \, \mathrm{d}y \, \mathrm{d}x\)

\item
  \(\int_0^{\pi/4} \int_{\sin x}^{\cos x} \sqrt{x + y^2} \, \mathrm{d}y\,
  \mathrm{d}x= \int_0^1 \int_{\arcsin y}^{\arccos y} \sqrt{x + y^2} \,
  \mathrm{d}x\, \mathrm{d}y\)

\item
  \(\int_{-1}^{1} \int_{0}^{1} e^{x^2 + y^2} \sin y \, \mathrm{d}x\,
  \mathrm{d}y= 0\)
\end{enumerate}

\end{document}
